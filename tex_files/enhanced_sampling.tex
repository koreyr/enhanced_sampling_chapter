\documentclass{book}

\begin{document}

\title{Enhanced Sampling Techniques}
\author{Jayakrishna Koneru and Korey Reid, and Paul Robustelli}
\date{\today}

\maketitle

\tableofcontents

\chapter{}
{Introduction}
% Your content goes here
Studying and understanding the underlying molecular dynamics of biological systems is very helpful. Very many experimental techniques are developed over the course of scientific history.
However, the there are limitations to experimental techniques for example we can't explain the atomistic details of molecular interactions, it will hard to extract the conformational dynamics
with fast transitions between them, hard to understand the localized dynamics of a biological system etc. From the dawn of molecular computational techniques it opened a new frontier to study
and explore the atomistic regiem of molecular interactions with the use of computational techniques. From simulating a small molecule containing a couple of atoms to large biological systems 
was made possible with integration of experimental data and optimizing the base parameters used in the simulations because of which we are able to reproduce physical relevant ensembles similar 
to experimental observables. Development of computational techniques and advancement in computational 
resources made it possible to study the bio-molecular systems of various length scales and there dynamics at atomistic level. Still computational techniques are limited by various factors
such as athenticity of the force field used, the time scale of the simulation, the size of the system, the sampling of the conformational space etc. To overcome some of these limitations
computational techniques widely know as enhanced sampling methods are developed which are helpful in reducing the simulation time scales required to generate the conformational ensembles.
There are many techniques in the literature in which we will be discussing about an method called solvent-scaled Replica Exchange with Solute Scaling which is an optimized version of 
Replica Exchange with Solute Scaling(REST2) method.        
  
\chapter{Theory} 

Hamiltonian representing the potential energy of the system can be written as sum of ots respective contributions as shown below :

\begin{center}
    \begin{equation}
        E_{n}^{scaled}(X_{n}) = \lambda_{n}^{pp} E_{pp}(X_{n}) + \lambda_{n}^{pw} E_{pw}(X_{n}) + \lambda_{n}^{ww} E_{ww} (X_{n})
    \end{equation}
\end{center}


Where, $\lambda_{n}$ is the scaling factor which scales the respective energy contributions as shown below :  

\begin{center}
    \begin{equation}
        \lambda_{n}^{pp}=\frac{\beta_{n}}{\beta_{0}} \ ; \quad \lambda_{n}^{pw}=\sqrt{\frac{\beta_{n}}{\beta_{0}}} \ ; \quad \lambda_{n}^{ww}=1;        
    \end{equation}
\end{center}

where $\beta_{0}=\frac{1}{k_{B}T_{0}}$.

\begin{center}
    \begin{equation}
        \epsilon_{n}^{rescaled} = \kappa_{n}^{2} * \epsilon_{n}
    \end{equation}
\end{center}

\begin{center}
    \begin{equation}
        \kappa_{n}=\kappa_{low}*\exp{ \biggl(n*\frac{\log(\kappa_{high}/\kappa_{low})}{N_{r}-1}\biggr)} \ ; \quad 1.00 \leq \kappa_{n} \leq 1.10
    \end{equation}
\end{center}

\begin{center}
    \begin{equation}
        \epsilon_{p:OW}=(\epsilon_{p:p}^{rescaled} * \epsilon_{OW:OW}^{rescaled})^{\frac{1}{2}}=\lambda_{n}^{pw}\kappa_{n}*(\epsilon_{p:p} * \epsilon_{OW:OW})^{\frac{1}{2}}
    \end{equation}
\end{center}

\begin{table}[h!]
\centering
\begin{tabular}{|c|c|c|c|c|c|}
\hline
Method & $T_{max}$K & $\lambda$ & $\epsilon_{CA}$ & $\kappa$ & $\epsilon_{OW}$ \\
\hline
-- & 300 & 1.0 & 0.359824 & 1.0 & 0.998989 \\
\hline
REST2 & 450 & 0.666667 & 0.239883 & 1.0 & 0.998989 \\
\hline
ssREST3 & 450 & 0.666667 & 0.239883 & 1.1 & 1.20878 \\

\hline
\end{tabular}
\caption{Table showing the differences and similarities of $\epsilon$ scaling between the REST2 and ssREST3 methods. In case of ssREST3 the 
water $\epsilon$ gets scaled along with the solute $\epsilon$ by a factor if $\kappa^{2}$ where as solvent parameters are not scaled during REST2.}
\label{tab:eps_table}
\end{table}

\end{document}