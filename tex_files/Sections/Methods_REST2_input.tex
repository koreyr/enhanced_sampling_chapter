
Implementing REST(2 or 3) using the GROMACS\cite{VanDerSpoel2005} molecular dynamics engine requires installation of PLUMED2 plugin version$\ge2.8$ and patching of GROMACS source code before compiling mdrun. Please note that PLUMED2 can be used alongside the AMBER MD engine version$ge18$ and NAMD version$\ge2.12$ at a significant cost to performance. For a simple walkthrough of this installation procedure for GROMACS please refer to Note \ref{notes:installation}.
%We need to discuss REST2 generation first, then state added edits to produce ssREST3
The first step in the ssREST3 implementation is to generate the a processed topology file your solvated protein of interest using \textit{-pp} option in \textit{gmx grompp}, note any position restraints contanted in an mdp file will be contained in the processed topology file.

\begin{lstlisting}[language=sh, basicstyle=\ttfamily\small, breaklines=true]
    gmx grompp -f *.mdp -c *.gro -r *.gro -p topol.top -o *.tpr -pp
\end{lstlisting}

\noindent This topology files will be used as to implement REST2 scaling procedure using PLUMED2.8 where one will supply the processed topology file and the corresponding $\lambda$ value as such :

\begin{lstlisting}[language=sh, basicstyle=\ttfamily\small, breaklines=true,escapeinside={(*}{*)}]
    plumed partial_tempering   (*$\lambda_{n}$*) < processed.top > scaled.top
\end{lstlisting}

After the topology files are generated for the respective replicates, the oxygen atom 
of the solvent, for example OW$_{tip4pd}$ which is the water oxygen atom of amber99sb-\textit{disp} force field, will have $\epsilon$ scaled by a factor of $\kappa^{2}_{n}$ to satisfy the scaling condition, 
 Eqs. \ref{eq:kappa_scaling} and \ref{eq:wat_scaling}. 
If $\kappa_{n}$ is set to 1.0 across all replicas the topologies conform to the REST2 convention. 
Along with the scaling of the water oxygen, we recover the interactions between the solvent molecules, and between solvent and ions, thus targeting only the protein-water LJ potential. 
This is accomplished by adding three non-bonded overrides to the \textit{[nonbonded]} section of the GROMACS topology file as shown in Table \ref{tab:eps_noscale_table}.
Once these changes are made to the topology, we are ready to simulate ssREST3.
Using the below command all the replicates are run in parallel and the the conformational exchange between the replicas is set for every 800 steps which equates to 1.6 ps.

\begin{lstlisting}[language=sh, basicstyle=\ttfamily\small, breaklines=true]
    gmx  grompp -f *.mdp -c *.gro -r *.gro -p scaled.top -o scaled.tpr
    gmx mdrun -s scaled.tpr -multi <replica folders> -replex 800 -deffnm replica -plumed plumed.dat
\end{lstlisting}


At the start of the simulations it is best practice to check the acceptance ratios between replicas to ensure that the scaling is not too aggressive nor too weak. 
A minimum of 20\% acceptance ratio is recommended for the simulations to be considered valid.