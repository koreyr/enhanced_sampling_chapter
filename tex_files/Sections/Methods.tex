\lstset{
    tabsize=4, % tab space width
    showstringspaces=false, % don't mark spaces in strings
    commentstyle=\color{green}, % comment color
    keywordstyle=\color{blue}, % keyword color
    stringstyle=\color{red} % string color
}

Implementing the ssREST3 in molecular dynamics simulations requires the use of the PLUMED2.8 plugin for GROMACS version 2023.5. 
The first step in the ssREST3 implementation is to generate the a processed topology file your solvated protein of interest using \textit{-pp} option in \textit{gmx grompp}.

\begin{lstlisting}[language=sh, basicstyle=\ttfamily\small]
    gmx grompp -f *.mdp -c *.gro -r *.gro -p topol.top -o *.tpr -pp
\end{lstlisting}

This topology files will be used as to implement REST2 scaling procedure using PLUMED2.8 where one will supply the processed topology file and the corresponding $\lambda$ value as such :

\begin{lstlisting}[language=sh, basicstyle=\ttfamily\small]
    plumed partial\_tempering  $\lambda_{n}$ < processed.top > scaled.top
\end{lstlisting}

After the topology files are generated for the respective replicates the heavy atoms 
of the solvent, for example OW$_{tip4pd}$ which is the water oxygen atom of amber99sb-\textit{disp} force field whose $\epsilon$ will be scaled by a factor of $\kappa^{2}_{n}$ to satisfy the scaling condition 
as mentioned in Eq. \ref{eq:kappa_scaling}, \ref{eq:wat_scaling}. 
If $\kappa_{n}$ is set to 1.0 across all replicas one can observe the it behaves like a conventional REST2. 
Along with the scaling of the water oxygen we also made sure that the self interaction of the solvent molecules and there interactions with the ions will not be effected by adding three additions lines in the \textit{[nonbonded]} section of the GROMACS topology file as shown in \ref{tab:eps_noscale_table}.
Once this changes are made to the topology files we are ready to run the simulations.
Using the below command all the replicates are run in parallel and the the conformational exchange between the replicas is set for every 800 steps which equates to 1.6 ps.

\begin{lstlisting}[language=sh, basicstyle=\ttfamily\small]
    gmx  grompp -f *.mdp -c *.gro -r *.gro -p scaled.top -o scaled.tpr
    gmx mdrun -s scaled.tpr -multi <replica folders> -replex 800 -deffnm replica -plumed plumed.dat
\end{lstlisting}


At the start of the simulations it is always a best practice to check the acceptance ratio between the replicas to ensure that the scaling is not too aggressive or too weak. 
A minimum of 20\% acceptance ratio is recommended for the simulations to be considered valid.