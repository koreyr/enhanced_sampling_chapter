We use C-terminal of $\alpha$-synuclein in presence of a ligand called "Fasudil" and use REST2 and REST3 enhanced sampling techniques to simulate with our base temperature at 300K 
and 450K as maximum temperature to generate the temperature ladder.
We made sure that the terminal capping is done for the 20 amino acid peptide.
We generate 10 replicates for REST2 and 8 replicates for REST3 simulations.
The protein is solvated in a 6.5 nm cubic box using tip4p-\textit{disp} a 4 point water model which has shown good agreement with IDP's and sufficient amount of NA$^{+}$ or CL$^{-}$ ions are added to netralize the 
system.
The system is energy minimized up to 5000 steps using steepest decent algorithm and then system has undergone a rigorous equilibration.
After a successful minimization the system is heated up to 300K under NVT ensemble for 500ps coupled to a temperature bath using V-rescale algorithm.
Once the solvated system is heated up, the system undergoes a 100ps equilibration under NPT ensemble where it is coupled with Berendsen barostat so that the pressure of the system 
is maintained at 1 bar.
Berendsen barostat helps in a quick convergence of pressure towards 1 bar after which we equilibrate under NPT condition for another 40ns using Parrinello-Rahman barostat which 
helped the system to achieve a convergence around 1 bar of pressure.
After the pressure was well converged the last frames of the replicas are used as a staring configurations for REST2 and REST3. 
The replicates are simulated under NVT ensemble for 1.98$\mu$s and 1.3$\mu$s totaling an aggregate of 19.8$\mu$s and 10.4$\mu$s for REST2 and REST3 ensembles.
Radius of Gyration which gives us an estimate whether the protein is collapsed or elongated is used as a metric and its distribution is computed using MDTraj after the trajectories 
are corrected for there periodicity.
As shown in the Fig : we can see that the protein ensemble simulated using REST3 stays elongated when compared to REST2 as there is a positive shift in the distribution of 
$\alpha$-carbon Rg. 

