Studying and critically understanding the underlying dynamics of biological systems at the atomistic scale can provide valuable insight into molecular mechanisms. 
Over many decades experimental techniques have been and continue to be developed in a collective effort to investigate processes at the molecular scale.
However, the limitations of experimental techniques often are restricted from producing clear and relevant explaintions at the atomistic level. 
This is in large part to the ensemble characteristic of experimental measurements and lends to the difficulty of extracting conformational dynamics at various timescales, as well as extracting localized dynamics of say a biological system. 
Upon entry of molecular dynamics the usefullness was apparent, albiet limited by computational speed. 
As molecular dynamics engines\cite{Weiner1981,gotz2012,salomon-ferrer2013,Brooks1983,Brooks2009,VanDerSpoel2005} and their accompanied forcefields\cite{Huang2016,Ploetz2021,Cornell1995,lindorff-larsen2010,Robustelli2018,Jakobsen2015,Piana2020} have matured, our ability to reach microsecond timescales has become routine. 
While the product of many decades of effort have resulted in our ability to simulate microseconds for a single biomolecule, vastly extending our comprehension of atomistic dynamics when paired with experimental results, these timescales are in fact beneath the threshold required to study rare events, e.g. allosteric transitions, or sample large degrees of freedom often accompanying IDPs. 

When the desire is to study conformational dynamics it becomes a neccessity for long-timescale simulations on supercomputing systems\cite{Shaw2009,Shaw2014}, for all other cases where statistical measures of an ensemble enhanced sampling techniques are available\cite{Lee2016,Wang2011,Qi2018,Vitalis2009,Zhang2023,Ray2023,Prakash2018}. 
%We need to append a reference to using enhanced sampling to produce starting conformations for unbiased simulation series plues references to others have done so. 
The phenomenological observation desired dictates which simulation methodology will provide usefull information.
One such method, Replica Exchange Solute Tempering or REST\cite{Liu2005,Wang2011,Zhang2023}, has arrisen from Hamiltonian Replica Exchange Molecular Dynamics (HREMD) a derivative of Replica Exchange Molecular Dynamics (REMD). 
For each of these methods, multiple parallel simulations are conducted in parallel. 
These parallel simulations undergo exchanges at a designated interval.
The relevance differences in each exchange method is discussed in the Introduction and Theory section, for a detailed account please refer to the original publications\cite{Sugita1999,Liu2005,Wang2011}.
 