From Replica Exchange Molecular Dynamics\cite{Sugita1999}, the hamiltonian representing the potential energy of the system can be written as sum of respective contributions, separated into protein-protein, protein-water and water-water:
\begin{center}
    \begin{equation}
        E_{n}^{REMD}(X_{n}) = \lambda_{n}^{pp} E_{pp}(X_{n}) + \lambda_{n}^{pw} E_{pw}(X_{n}) + \lambda_{n}^{ww} E_{ww} (X_{n})
    \label{eq:remd_hamiltonian}
    \end{equation}
\end{center}
$\lambda_{n}^M$ is the scaling factor, where $M=\{ pp, pw, ww\}
$ which scales the corresponding energy term. For REST2\cite{Wang2011}, $\lambda_n^{ww}=1$, $\lambda_n^{pp}=(\lambda_n^{pw})^2=\lambda_n$, for simplicity the REST2 hamiltonian simplifies to:
% import simplified hamiltonian of REST2
\begin{center}
    \begin{equation}
        E_{n}^{REST2}(X_{n}) = \lambda_{n} E_{pp}(X_{n}) + \sqrt{\lambda_{n}} E_{pw}(X_{n}) +  E_{ww} (X_{n})
        \label{eq:rest2_hamiltonian}
    \end{equation}
\end{center}
% import lambda scaling equation
where,
\begin{center}
    \begin{equation}
        % \lambda_{n}^{pp}=\frac{\beta_{n}}{\beta_{0}} \ ; \quad \lambda_{n}^{pw}=\sqrt{\frac{\beta_{n}}{\beta_{0}}} \ ; \quad \lambda_{n}^{ww}=1;   
        \lambda_n = \frac{\beta_{n}}{\beta_{0}}
    \label{eq:lambda_scaling}     
    \end{equation}
\end{center}
% import temperature factor expression
and $\beta_{n}=\frac{1}{k_{B}T_{n}}\ for\ n=\{0,1,2,\ldots,n_{replica}\}$. 

Upon investigation, disordered proteins containing hydrophobic residues undergo conformational collapse with respect to scaling  $E^{pw}$ to higher effective temperatures. This outcome is unfavorable when attempting to capture a representative ensemble as hydrophobic collapse reduces the overall sampling of the proteins degrees of freedom. % We need to say something here to the effect: REST2's ability to enhance hydrophobic interactions is synonamous with reducing the quality of the solvent, water. 
\citeauthor{Zhang2023} \citeyear{Zhang2023} provided a basis for biasing the scaling such that protein collapse is minimized or negated. The formalism they proposed,



% \begin{center}
%     \begin{equation}
      $  \epsilon_{i}^{scaled}=\lambda_{n}\cdot \epsilon_i$ and $\epsilon_{OW}^{scaled} = \kappa_{n}^{2} * \epsilon_{OW}$
    % \label{eq:epsilon_scaling}
%     \end{equation}
% \end{center}

\begin{center}
    \begin{equation}
        \kappa_{n}=\kappa_{low}*\exp{ \biggl(n*\frac{\log(\kappa_{high}/\kappa_{low})}{N_{r}-1}\biggr)} \ ; \quad 1.00 \leq \kappa_{n} \leq 1.10.
    \label{eq:kappa_scaling}
    \end{equation}
\end{center}



\begin{center}
    \begin{equation}
        \epsilon_{p:OW}=(\epsilon_{p:p}^{rescaled} * \epsilon_{OW:OW}^{rescaled})^{\frac{1}{2}}=\lambda_{n}^{pw}\kappa_{n}*(\epsilon_{p:p} * \epsilon_{OW:OW})^{\frac{1}{2}}
    \label{eq:wat_scaling}
    \end{equation}
\end{center}